% Conclusions provide a clear and thorough summary, with interesting future work suggested.

% Thoughtful personal reflection on the lessons learned.

\label{sec:5}
This project has exceeded my original success criteria: I have created a distributed visual SLAM system that performs better than comparable state-of-the-art systems and demonstrated real-world performance by deploying it on to physical robots. This is facilitated by a unique approach to the distributed pose graph estimation problem and the addition of features such as relocalization and long-term map reassociation.

In addition, I have developed a full suite of novel infrastructure including visualization, simulation, and evaluation tools, largely due to a lack of existing libraries for the relatively new field of multi-agent SLAM. All these tools, along with my core system, are made open-source with compiled Docker containers also provided.

\section{Future Work}
\label{sec:future-work}
\textbf{Deep neural network feature extraction and matching.} In recent years there has been an exciting emergence of neural network based feature descriptors. SuperPoint \autocite{detone18superpoint} is an example of such a descriptor, providing impressively robust results. In addition, feature matchers such as LightGlue \autocite{lindenberger2023lightglue} are able to learn the geometric regularities of camera transformations through an end-to-end neural network training process to robustly match features across images, removing the need for the RANSAC \texttt{+} epipolar geometry model feature matching described in the \nameref{sec:visual-slam-visual-odometry} section. Leveraging such technologies in a multi-agent SLAM system may yield improved accuracy and robustness, especially in dynamic environments.

\textbf{Real-world deployment of multi-agent research.} Many interesting systems have been presented in the field of multi-agent robotics, such as reinforcement learning based path planning \autocite{bettini2022vmas} and collaborative aerial 3D printing \autocite{zhang2022aerial} and bridge construction \autocite{1d3d53ca-43c1-3e44-8ce1-3b0c374a8f1e}. These systems have all been demonstrated in motion capture labs, however decentralized SLAM systems such as mine present a pathway towards their full-scale deployment in practical settings.

\textbf{Bandwidth optimization.} Ad-hoc mesh networks are necessary for the real-world deployment of distributed systems, however, their network capacity is often limited and some systems prefer broadcast messages over peer-to-peer. It is worth exploring how the data transfer characteristics of my system can be improved to better suit such a networking environment.

\section{Lessons Learned}
\label{sec:lessons-learned}
The most important lesson learned from this project is that cutting-edge research does not necessarily equate to well-written or documented code. I chose to use the single-agent ORB-SLAM3 system as the foundation of my project due to it being well-cited and widely regarded as the most accurate system available.


However, I quickly discovered that it was extremely hard to build upon their codebase. Large chunks of code were commented out without explanation, dataflow was hard to decipher, and it ran across four threads with poor concurrency control. Due to the above complexities, as well as it being my first time using C++, it took almost a month to get ORB-SLAM running, and many more before I began making good progress on the distributed aspect of the project – far longer than I had expected. In summary, I should have chosen a system based on the code quality and support from the researchers, as opposed to purely its performance.

%This was quite frustrating, and required me to spend more time on this project than was probably reasonable

% \section{Reflection}
% \label{sec:reflection}

% Four years ago when I was appling to Cambridge, I wrote in my personal statement that I was facinated by computer vision – specifically visual SLAM. I find it rather fitting that I now conclude my degree by developing a novel distributed visual SLAM system that performs better than comparable state-of-the-art systems – something I could have only dreamt of when writing my personal statement all those years ago.

% Overall, I am very satisfied with the results achieved by my system. I knew from the beginning that I did not want to spend a year creating something worse than what was already out there, and I am proud to have acheived that.