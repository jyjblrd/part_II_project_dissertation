% Conclusions provide a clear and thorough summary, with interesting future work suggested.

% Thoughtful personal reflection on the lessons learned.

\label{sec:5}

\section{Future Work}
\label{sec:future-work}
Visual SLAM is a mature field, with improvements being incremental as opposed to ground breaking.

\section{Lessons Learned}
\label{sec:lessons-learned}
The most important lesson learnt from this project is that cutting edge research does not equate to well written or documented code. I committed to using the single agent ORB-SLAM3 system as the foundation for my project, however I quickly discovered that it was extremely hard to build upon their codebase. Large chunks of code were commented out without explination, everything was defined as an instance variable making dataflow hard to decypher, and many function were named \texttt{localMapping()}, \texttt{localMapping2()}, \texttt{localMapping3()}, with different parts of the code arbitrarliy choosing any one of the functions to use. In addition, there was simply a large amount of inherit complexity within the codebase as it is the most advanced single-agent SLAM system and is multithreaded.

ORB-SLAM also did not compile out-of-the-box, requiring me to make modifications to both the makefiles and codebase. Due to the above complexities as well as being my first time using C++, it took almost month to get ORB-SLAM running and many more before I began making good progress on the distributed aspect of the project – far longer than I had expected. In summary, I should not have assumed that research code would be easy to compile and build upon, even if it was one of the most well cited and performant systems in the field.

\section{Reflection}
\label{sec:reflection}

Four years ago when I was appling to Cambridge, I wrote in my personal statement that I was facinated by computer vision – specifically visual SLAM. I find it rather fitting that I now conclude my degree by developing a distributed visual SLAM system that performs better than comparable state of the art systems – something I could have only dreamt of when writing my personal statement all those years ago.

Overall, I am very satisfied with the results achieved by my system. I knew from the beginning that I did not want to spend a year creating something worse than what was already out there, however that is not an easy task in such a well researched field within computer science. Importantly, I chose to tackle a moderately niche yet well motivated area within the field, in this case a monocular video multi-agent system.