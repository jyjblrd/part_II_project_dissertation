% Compelling argument demonstrating success criteria met, or well-justified explanation of different direction taken.

% Excellent critical thought and interpretation which substantiate any claims of success.

\label{sec:4}

\section{Review of Success Criteria}
\label{sec:review-of-success-criteria}
My project has met all success criteria as outlined below:

\begin{enumerate}[font=\bfseries]
    \item[1a.]\textbf{Multiple agents will be able to localize themselves within a world using purely visual data.} \\
    My system is the first decentralized SLAM system capable of operating using only monocular camera data. This removes the need for large and expensive stereo camera, LiDAR or RGBD sensors.

    \item[1b.]\textbf{Agents will be capable of communicating with each other to build a shared understanding of the world.} \\
    Agents use ROS to perform decentralized and reliable communication with eachother, and are able to merge maps and share key frames to build a shared world.

    \item[1c.]\textbf{Agents will be able to act independently, failing gracefully if it loses communication with its peers.} \\
    Agents fall back to performing single-agent visual SLAM when communication with their peers is degraded or lost. Once communication is regained, the agents share their unsent map data.

    \item[2.]\textbf{Evaluate the capabilities of the system compared to a comparable single-agent system.} \\
    This is done in the \nameref{sec:benchmarking} Section. I go beyond the original success criteria by also comparing my system to comparable multi-agent systems, demonstrating my system's superior performance.

\end{enumerate}


After discussions with my supervisor, we decided to not pursue my original project extensions, instead focusing on deploying the SLAM system on physical robots and building... TODO

\section{Benchmarking}
\label{sec:benchmarking}
This section will benchmark the performance of my system when run on industry standard visual SLAM datasets. All evaluations are run using only monocular camera data.

The Central Management Interface is used to stream the dataset to the agents and my library \textit{Multi Agent EVO} is used to provide the Absolute Trajectory Error (ATE).

\subsection{EuRoC Machine Hall}
\label{sec:euroc-machine-hall}
The EuRoC Machine Hall dataset \autocite{burri2016euroc} is one of the most widely used visual SLAM datasets, providing a 752x480 20fps video feed and millimeter level ground truth data at 20hz. The camera rig is attached to a Micro Aerial Vehicle (MAV) which flies through a large machine room. To simulate a multi-agent dataset, we run the Machine Hall 01-03 scenarios in parallel on three agents.

All three agents merge initialize maps of vastly different scales but are able to merge within TODO seconds and begin sharing map data.

\autoref{fig:euroc-mh-01-03-line-plot} shows the ATE of my system when running this dataset. The overall RMS ATE is only 6.2cm over the 279m total trajectory length, demonstrating my system's very impressive accuracy. It is clear that my system is performing SLAM as opposed to simple visual odometry as the ATE returns back to the baseline at the end of each run when the agents return back to their starting position.

\begin{figure}[h]
    \centering
    \includegraphics[width=\linewidth]{figures/EuRoC_MH_01-03_line_plot.pdf}

    \caption{Plot of my system's average trajectory error (ATE) with respect to the ground truth when running the EuRoC Machine Hall 01-03 scenarios in parallel on three agents. The average RMSE of the multi-agent system is 0.062m over the 279m total trajectory length. \captionbreak RMSE represents the root mean square of the ATE.}
    \label{fig:euroc-mh-01-03-line-plot}
\end{figure}

TODO do a plot of like 5 runs to show low variance.

We now run the Machine Hall 01 and 02 scenarios on two agents to analyze the network usage in \autoref{fig:euroc-mh-01-02-bandwith}. Initially, the agents send bag of word information before quickly detecing a merge opportunity. Agent 1 then procedes to send its full map to agent 2, which can be seen as a large initial spike in network bandwith. The agents successfully merge, and begin exchanging key frames. The rate of key frame data being sent rises and falls depending on how much new area the agent is exploring.

Along with sending key frames, the agents sporadically send map points to refine their map alignment. This occurs less frequently the longer system runs due to the additive increase multiplicative decrease method used to schedule map alignments, described in the \nameref{sec:map-alignment-refiner} Section. However, the size of the messages also grows over time due to the larger number of tracked map points.

The total data exchanged by the two agent system is 55.7MB over 184 seconds, giving an average communication rate of 302.6KB/s. 84\% of this is from key frame data, which is sent at an average of 127KB/s by each agent.

\begin{figure}[h]
    \centering
    \begin{subfigure}[b]{0.55\linewidth}
        \centering
        \marginbox{0 0 0 0} {
            \includegraphics[width=\linewidth, valign=t]{figures/apr11_mh_trajectory_b_bandwith.pdf}
        }
        \caption{Total system data over time}
    \end{subfigure}%
    ~
    \begin{subfigure}[b]{0.45\linewidth}
        \flushright
        \adjustbox{valign=t, width=\linewidth}{
            \marginbox{0.2in 0 0 0} {

                \def\arraystretch{1.2}
                \begin{tabular}{ |c|l|r|r| }
                    \cline{3-4}
                    \multicolumn{2}{}{}                       & \multicolumn{1}{|c|}{KB} & \multicolumn{1}{|c|}{Avg. KB/s}         \\
                    \hline
                    \multirow{3}{*}{Key Frames}               & $agent_0$                & 69,971                          & 351.9 \\
                                                              & $agent_1$                & 63,908                          & 321.4 \\
                                                              & $agent_2$                & 65,164                          & 327.7 \\
                    \hline
                    \multirow{3}{*}{BoWs}                     & $agent_0$                & 371                             & 1.9   \\
                                                              & $agent_1$                & 437                             & 2.2   \\
                                                              & $agent_2$                & 1,496                           & 7.5   \\
                    \hline
                    \multirow{2}{*}{Full Map}
                                                              & $agent_{0\to1}$          & 13,953                          & 70.2  \\
                                                              & $agent_{0\to2}$          & 7,319                           & 36.8  \\
                    \hline
                    \multicolumn{2}{|c|}{Alignment Data}      & 22,560                   & 113.6                                   \\
                    \hline
                    \multicolumn{2}{|c|}{\textbf{Total Data}} & \textbf{245,218}         & \textbf{1,233.1}                        \\
                    \hline
                \end{tabular}
            }
        }
        \caption{Total system data by message type}
        \vfill

        \adjustbox{valign=b, width=\linewidth}{
            \marginbox{0.2in 0 0 0.3in} {
                \def\arraystretch{1.2}
                \begin{tabular}{ |l|r|r|r|r| }
                    \cline{2-5}
                    \multicolumn{1}{}{} & \multicolumn{2}{|c|}{Sent} & \multicolumn{2}{|c|}{Received}                                                               \\
                    \cline{2-5}
                    \multicolumn{1}{}{} & \multicolumn{1}{|c|}{KB}   & \multicolumn{1}{|c|}{Avg. KB/s} & \multicolumn{1}{|c|}{KB} & \multicolumn{1}{|c|}{Avg. KB/s} \\
                    \hline
                    $agent_0$           & \textbf{114,585}           & \textbf{576.2}                  & \textbf{131,005}         & \textbf{658.8}                  \\
                    \hline
                    $agent_1$           & \textbf{64,345}            & \textbf{323.6}                  & \textbf{173,554}         & \textbf{872.8}                  \\
                    \hline
                    $agent_2$           & \textbf{66,660}            & \textbf{335.2}                  & \textbf{164,606}         & \textbf{827.7}                  \\
                    \hline
                \end{tabular}
            }
        }
        \caption{Data by agent}
    \end{subfigure}%

    \caption{Bandwidth used by the EuRoC Machine Hall 01-03 scenarios running on my SLAM system.}
    \label{fig:euroc-mh-01-02-bandwith}
\end{figure}

\subsection{TUM-VI Rooms}
\label{sec:tum-rooms}
The TUM visual-intertial dataset \autocite{8593419} consists of handheld fisheye 512x512 video with ground truth data. As before, we use only monocular visual data and combine the room1-3 sessions to simulate a multi-agent dataset. The "room" environment is used for this evaluation, which is a motion capture lab with plain flat walls with some posters hung up. There is less texture in this environment than the machine hall, making visual-only SLAM more difficult. Additionally, parts of the room are revisited dozens of times by different agents, allowing us to evaluate our system's ability to relocalize agents within previously mapped environments.


\begin{figure}[h]
    \centering
    \includegraphics[width=\linewidth]{figures/apr11_tum_room_trajectory_a_line_plot.pdf}

    \caption{Plot of my system's average trajectory error (ATE) with respect to the ground truth when running the EuRoC Machine Hall 01-03 scenarios in parallel on three agents. The average RMSE of the multi-agent system is 0.062m over the 279m total trajectory length. \captionbreak RMSE represents the root mean square of the ATE.}
    \label{fig:euroc-mh-01-03-line-plot}
\end{figure}

\begin{figure}[h]
    \centering
    \begin{subfigure}[b]{0.55\linewidth}
        \centering
        \marginbox{0 0 0 0} {
            \includegraphics[width=\linewidth, valign=t]{figures/apr11_tum_room_trajectory_a_bandwith.pdf}
        }
        \caption{Total system data over time}
    \end{subfigure}%
    ~
    \begin{subfigure}[b]{0.45\linewidth}
        \flushright
        \adjustbox{valign=t, width=\linewidth}{
            \marginbox{0.2in 0 0 0} {

                \def\arraystretch{1.2}
                \begin{tabular}{ |c|l|r|r| }
                    \cline{3-4}
                    \multicolumn{2}{}{}                       & \multicolumn{1}{|c|}{KB} & \multicolumn{1}{|c|}{Avg. KB/s}         \\
                    \hline
                    \multirow{3}{*}{Key Frames}               & $agent_0$                & 37,599                          & 264.9 \\
                                                              & $agent_1$                & 33,066                          & 233.0 \\
                                                              & $agent_2$                & 29,963                          & 211.1 \\
                    \hline
                    \multirow{3}{*}{BoWs}                     & $agent_0$                & 217                             & 1.5   \\
                                                              & $agent_1$                & 188                             & 1.3   \\
                                                              & $agent_2$                & 426                             & 3.0   \\
                    \hline
                    \multirow{2}{*}{Full Map}
                                                              & $agent_{0\to1}$          & 1,472                           & 10.4  \\
                                                              & $agent_{0\to2}$          & 2,631                           & 18.5  \\
                    \hline
                    \multicolumn{2}{|c|}{Alignment Data}      & 6,947                    & 49.0                                    \\
                    \hline
                    \multicolumn{2}{|c|}{\textbf{Total Data}} & \textbf{112,511}         & \textbf{792.8}                          \\
                    \hline
                \end{tabular}
            }
        }
        \caption{Total system data by message type}
        \vfill

        \adjustbox{valign=b, width=\linewidth}{
            \marginbox{0.2in 0 0 0.3in} {
                \def\arraystretch{1.2}
                \begin{tabular}{ |l|r|r|r|r| }
                    \cline{2-5}
                    \multicolumn{1}{}{} & \multicolumn{2}{|c|}{Sent} & \multicolumn{2}{|c|}{Received}                                                               \\
                    \cline{2-5}
                    \multicolumn{1}{}{} & \multicolumn{1}{|c|}{KB}   & \multicolumn{1}{|c|}{Avg. KB/s} & \multicolumn{1}{|c|}{KB} & \multicolumn{1}{|c|}{Avg. KB/s} \\
                    \hline
                    $agent_0$           & \textbf{49,083}            & \textbf{345.9}                  & \textbf{63,644}          & \textbf{448.5}                  \\
                    \hline
                    $agent_1$           & \textbf{33,254}            & \textbf{234.3}                  & \textbf{76,624}          & \textbf{539.9}                  \\
                    \hline
                    $agent_2$           & \textbf{30,389}            & \textbf{214.1}                  & \textbf{80,649}          & \textbf{568.3}                  \\
                    \hline
                \end{tabular}
            }
        }
        \caption{Data by agent}
    \end{subfigure}%

    \caption{Bandwidth used by the TUM-VI Rooms 01-03 scenarios running on my SLAM system.}
    \label{fig:euroc-mh-01-02-bandwith}
\end{figure}

%evaluate how

\section{Comparison to Related Work}
\label{sec:comparison-to-related-work}

Direct comparisons with other multi-agent SLAM systems is difficult since many of them rely on stero cameras, IMUs, and LiDAR sensors. However, this section attempts to compare my system to the latest research in the field to give context to its performance relative to other state of the art systems.

\subsection{CCM-SLAM}
\label{sec:ccm-slam}
Centralized Collaborative Moncular SLAM (CCM-SLAM) (2019) from the ETH Zurich Vision for Robotics Lab \autocite{schmuck2019ccm} is the most comparable multi-agent SLAM system in recent literature, and is well cited in the field. Their approach also uses purely monocular vision with ORB-SLAM to the front-end tracking and local mapping stages of the SLAM pipeline, with a custom backend to integrate data from multiple agents into the shared map.

The results of CCM-SLAM and my system running the EuRoC Machine Hall 01-03 dataset are presented in \autoref{fig:comparison-to-multi-agent-systems}. \textbf{It is important to note that CCM-SLAM is a centralized system, significantly simplifying their multi-agent SLAM problem compared to my distributed system.} Nevertheless, my system is able to outperform CCM-SLAM by TODO\% in terms of RMS ATE, demonstrating its compeditiveness. In addition, my system's total data transfered comparable to CCM-SLAM, being only TODO\% higher.

\subsection{VINS-Mono Multisession SLAM}
VINS-Mono (2018) \autocite{8421746} is a popular open source single agent monocular vision SLAM system. While it is a single agent system, it additionally has multisession capabilities. This allows multiple datasets to be played consecutively, with each session building upon the map built by the previous sessions (eg. session 2 is initialized in the map generated after running session 1). This is an easier problem than multi-agent SLAM, where the sessions are run in parallel and the agents have to share their maps in real time.As presented in \autoref{fig:comparison-to-multi-agent-systems}, my system still outperforms VINS-Mono multisession by TODO\%.

\begin{figure}[h]
    \centering

    \caption{Comparison of my SLAM system to comparable state of the art multi-agent systems.}
    \label{fig:comparison-to-multi-agent-systems}
\end{figure}

\subsection{Comparison to Single-Agent SLAM Systems}
Along with providing relative positioning, we would expect a multi-agent SLAM system to have higher accuracy than a comparable single-agent system. \autoref{fig:comparison-to-single-agent-systems} presents my multi-agent system's errors compared to ORB-SLAM3 and VINS-Mono running the datasets individually. ORB-SLAM3 is widely regarded as the most accurate single-agent SLAM system currently avaiable and was used to perform the tracking and local mapping in my multi-agent system, so there is no suprise that it performs very well. However, my system still outperforms the others, demonstrating the clear benifits of collaborative SLAM.

An observation is that some agents yield a greater improvment in error from the multi-agent system than others, which I hypothesize is the result of the varying ability amoung agents to utilize the shared map. For example, $agent_2$ has a similar RMS ATE in both the single-agent and multi-agent systems. This is due to it exploring an entirely different area to $agnet_0$ and $agent_1$, preventing it from benifiting from the data collected by the other agents.

\autoref{} visualizes the collaborative shared map built by my system compared to the individual maps from the single-agent ORB-SLAM3. There is clear deep integration between the different agents' map data in my collaborative map, giving the agents more numerous and accurate features to use when localizing themselves, resulting in lower trajectory errors.

\begin{figure}[h]
    \centering

    \caption{Comparison of my SLAM system to comparable state of the art single-agent systems.}
    \label{fig:comparison-to-single-agent-systems}
\end{figure}

TODO: talk about why we only compare with euroc MH



% compare data transfer with covins and ccm-slam which is shit

% Qualitative and quantitative


\section{Real World Experiments}
\label{sec:real-world-experiments}
Deploying my system on physical robots demonstrates its ability to be run in real time within the computational, bandwith and latency constraints of real world systems. Furthermore, it demonstrates the robustness of my development framework which allowed seamless migration from local testing using simulations to a real distributed system using live camera feeds with no changes to my code base.

The details of the real world setup are given in the \nameref{sec:real-world-implementation} Section. In summary, the Cambridge RoboMasters platform was used

\subsection{Multi-Agent Collision Avoidance}
\label{sec:multi-agent-collision-avoidance}
In this experiment, we showcase the NMPC collision avoidence motion controller working in conjunction with the SLAM system to avoid collisions with both static and dynamic obstacles. Real time relative position data from the SLAM system is fed to the NMPC collision avoidance motion controller which controls the robot's velocity accordingly. The SLAM system is running locally on each Cambridge RoboMaster, with communication faciltated by a router in the lab.

\autoref{fig:collision-avoidance} tests the system in an intersection environment, where two robots (traveling along the h orizontal and vertical axis) would normally collide. The shared map generated by the two agents allows them to localize each other even when their views do not overlap and they can not see the other agent, showcasing the versatility and wide ranging use cases of my distributed SLAM system. Additionally, this demonstrate the real time capabilities of my SLAM system, allowing the robots to react fast enough to avoid collisions in dynamic environments.

Out of the four consecutive trials run in this environment, there were zero collisions between the two agents, and \autoref{fig:collision-avoidance-distance-plot} shows that the distance between agents never went below the collision threshold of 0.55 meters.


\begin{figure}[h]
    \centering
    \includegraphics[width=\linewidth]{figures/mar25_1_tracer_graph.pdf}

    \caption{Demonstration of multi-agent collision avoidance, facilitated by my distributed SLAM system running locally on the Cambridge RoboMasters. Two robots are set 90° to each other in an intersection environment, with no direct view of the other robot and little visual overlap (right). The agent travelling along the Y axis is given a goal pose on the other side of the intersection, and successfully avoids a collision when the agent travelling along the X axis is pushed through the intersection. The trajectories generated by the SLAM system are presented on the left charts.}
    \label{fig:collision-avoidance}
\end{figure}

\begin{figure}[h]
    \centering
    \includegraphics[width=\linewidth]{figures/mar25_1_distance_plot.pdf}

    \caption{Plot of the ground truth distance between the two robots throughout all four collision avoidance trials conducted in the intersection environment. The dips between trials are the robots' positions being reset. \autoref{fig:collision-avoidance} visualizes Trial 4 in more detail.}
    \label{fig:collision-avoidance-distance-plot}
\end{figure}

\subsection{Augmented Reality Visualization}
\label{sec:augmented-reality-visualization}

My projected included the development of an augmented reality visualization system to view my SLAM system's data overlayed on top of a video captured by an external camera. \autoref{} shows a snapshot of this visualization, with the agents' map points and predicted locations overlayed on top of a video.