\documentclass[12pt,a4paper,twoside]{article}
\usepackage{preamble} % imported packages/macros/styling

\usepackage[sorting=none]{biblatex}
\usepackage[bottom]{footmisc}
\usepackage[none]{hyphenat}

\addbibresource{ref.bib}

\begin{document}
% ----------------------------------------------------------------------

\begin{titlepage} 

\vspace*{\fill}

% title page
\begin{center}
  \Huge
  \textbf{Distributed Visual Simultaneous \\ Localization and Mapping} \\[6mm]
  \Large
  Part II Computer Science Project Proposal \\[2mm]
  Joshua Bird, jyjb2 \\[2mm]
  \today \\[8mm]
\end{center}

\vspace{150pt}

{\large
\begin{tabular}{ll}
  \bf Project Originator:  & Joshua Bird, Jan Blumenkamp          \\[4mm]
  \bf Project Supervisor:  & Jan Blumenkamp                \\[4mm]
  \bf Director of Studies: & Ramsey Faragher                   \\[4mm]
  \bf Overseers:           & Jonathon Crowcroft, Paula Buttery
\end{tabular}
}

\vspace{170pt}

\end{titlepage}

\section{Introduction}

Visual simultaneous localization and mapping (Visual SLAM) is a technology used to localize the 6dof pose of a camera sensor while also creating a 3D map of the captured environment. Visual SLAM is often used in systems such as self-driving cars, augmented reality devices, and autonomous drones due to the low cost and weight of camera sensors when compared to the sensors used in other SLAM systems such LIDAR, RGB-D cameras or Radar.

While there have been many advanced implementations of visual SLAM over the last decade, very few of them have focused on distributed multi-agent systems, instead focusing on single-agent (e.g. ORB-SLAM3 \autocite{ORBSLAM3_TRO}) or centralized multi-agent systems (e.g. CCM-SLAM \autocite{schmuck2019ccm}). Distributed multi-agent visual SLAM systems have broad reaching use cases, as the lack of a centralized management server allows the system to be robust to failure and used in environments where network infrastrure may be lacking, such as autonomous drone swarms in search and rescue operations or collaboration amongst self-driving cars.

My project aims to implement a distributed multi-agent visual SLAM system, where agents are able to share information with each other to enable more accurate localization and collaboration in map building.

Key components of this project include:
\begin{enumerate}
  \item Create an environment for testing and evaluating a multi-agent visual SLAM system locally.
  \item Designing and implement a communication layer to enable agents to serialize/deserialize key datastructures (e.g. Keyframes, Map Points, etc.) and share information with each other.
  \item Ensure robust performance, even when connectivity between agents is degraded.
\end{enumerate}


Possible extensions to this project include:
\begin{enumerate}
  \item \label{extension:intelligentSharing} Enable agents to intelligently share information to their peers, only sending map fragments that are relevant.
  \item \label{extension:distributedComputation} Distributing calculations amongst the different agents, preventing the duplication of computationally intensive work such as global bundle adjustment. One possible approach would be to leverage gaussian belief propagation, a message passing algorithm to optimize a factor graph representation of the bundle adjustment problem.
  \item \label{extension:mapCompression} Map compression algorithms.
  \item \label{extension:extraSensors} Adding additional sensors such as ultra-wideband sensors to better localize drones using sensor fusion.
\end{enumerate}

\newpage 

\section{Starting Point}

Visual SLAM systems are a mature and well researched subfield of Computer Science, with many advanced implementations. To avoid spending the majority of my time re-implementing a visual SLAM system, I will instead be using an existing \textbf{single-agent} visual SLAM implementation as a starting point for my project. This will allow me to focus my efforts on the distributed aspect of my project, which I beleive is novel and under-researched in the field. Furthermore, by using a cutting edge single-agent SLAM system as a foundation for my project, I hope to be able to create a distributed SLAM system that is accurate and performant enough to have real-world use cases. 

Thus far, I have forked the ORB-SLAM3 git repository\footnote{ORB-SLAM3 is a cutting edge visual SLAM implementation published in 2021 by Campos et al.}\footnote{\url{https://github.com/UZ-SLAMLab/ORB_SLAM3}} \cite{ORBSLAM3_TRO} and made minor changes to the codebase to allow it to run on my machine.

I have no prior experience working with SLAM systems, but I have done research on the current state of multi-agent visual SLAM systems to evaluate the feesibility of my project, and to try prevent it being a duplication of past work.

\section{Evaluation}
\label{sec:Evaluation}
I will evaluate this project by comparing my multi-agent system to a comparable single-agent system when run on the same dataset. This will involve:
\begin{itemize}
  \item Quantitatively comparing the RMS error of the predicted trajectory to the ground truth.
  \item Qualitatively comparing the maps generated by both systems, particularly how much area they cover.
\end{itemize}

Additionally, I will evaluate any extensions (if applicable) as follows:
\begin{itemize}
  \item Extension~\ref{extension:intelligentSharing} \& \ref{extension:mapCompression}: Reduction to total bytes sent between nodes in the system.
  \item Extension~\ref{extension:distributedComputation}: Reduction in total computation done by multi-agent system.
  \item Extension~\ref{extension:extraSensors}: Improvement in localization performance.
\end{itemize}

To evaluate my system, I plan on using a combination of popular public datasets (e.g. EuRoC \autocite{doi:10.1177/0278364915620033}, KITTI \autocite{doi:10.1177/0278364913491297}) self-made datasets and datasets generated in the Cambridge Multi-Robot and Multi-Agent Systems Lab\footnote{\url{https://proroklab.org}}.

\section{Success Criteria}

For the project to be deemed a success, the following must be successfully completed:

\begin{enumerate}
  \item Create an implementation of a multi-agent visual SLAM system, with the following capabilities:
    \begin{enumerate}
      \item Multiple agents will be able to localize themselves within a world using purely visual data.
      \item Agents will be capable of communicating with each other to build a shared understanding of the world. 
      \item Agents will be able to act independently, failing gracefully if it loses communication with its peers.
    \end{enumerate}
  \item Evaluate the capabilities of the system, following the guidelines set out in \autoref{sec:Evaluation}
\end{enumerate}

\newpage 

\section{Timetable and Milestones}

\subsection*{Weeks 1 to 2 (16 October - 29 October)}
Get single-agent open source visual SLAM implementations running from the research code provided. Explore possible simulators and integrate them with the visual SLAM program.

\subsection*{Weeks 3 to 4 (30 October - 12 November)}
Continue preparatory work and set up a solid development environment for me to use when writing my own code. Includes figuring out how to develop and run a distributed system on my local machine.

\subsection*{Weeks 5 to 6 (13 November - 26 November)}
Create/find a dataset to test my distributed visual SLAM implementation on. 

Begin implementing distributed visual SLAM, with the different instances communicating with each other and sharing information.

\subsection*{Weeks 7 to 8 (27 November - 10 December)}
Continue implementing distributed visual SLAM system.

\subsection*{Weeks 9 to 11 (11 December - 31 December)}
Holiday \& revision.

\subsection*{Weeks 12 to 13 (1 January - 14 January)}
Holiday \& revision.

Continue implementing distributed visual SLAM system. 

\subsection*{Weeks 14 to 15 (15 January - 28 January)}
Explore implementing extension~\ref{extension:intelligentSharing} to intelligently share information with peers, only sending over useful parts of the map to them.

\subsection*{Weeks 16 to 17 (29 January - 11 February)}
Continue implementing distributed visual SLAM system and extension~\ref{extension:intelligentSharing}. 

Spend time on other additional goals if I have time. 

\subsection*{Weeks 18 to 19 (12 February - 25 February)}
Explore implementing extension~\ref{extension:distributedComputation} to prevent the duplication of work spent on doing global bundle adjustments.

\subsection*{Weeks 20 to 21 (26 February - 10 March)}
Begin evaluating project and writing dissertation.

\subsection*{Weeks 22 to 23 (11 March - 24 March)}
Continue evaluating project and writing dissertation.

\subsection*{Weeks 24 to 25 (25 March - 7 April)}
Submit draft of dissertation to supervisor and DOS and make improvements.

Set aside time for exam revision.

\subsection*{Weeks 26 to 27 (8 April - 21 April)}
Finalize dissertation and get any final feedback.

Set aside time for exam revision.

\subsection*{Weeks 28 - 29 (22 April - 5 May)}
Padding for any final work to be done.

Set aside time for exam revision.

\subsection*{Week 30 (10 May)}
Dissertation Deadline

\section{Resource Declaration}

I will be using my personal laptop (MacBook Air M2) as my primary machine for software development, along with Git to continuously back up my code and dissertation to the cloud in the event of any hardware failure. \textit{I accept full responsibility for this machine and I have made contingency plans to protect myself against hardware and/or software failure.}

I have asked for permission to use the Robotics Lab whenever it is free, as it does not have a booking system. This is not a critical resource, as I have multiple dataset sources as noted in \autoref{sec:Evaluation}.

% ----------------------------------------------------------------------

\newpage

\printbibliography[heading=subbibliography]

\appendix

\end{document}